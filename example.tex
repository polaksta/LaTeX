\documentclass[aspectratio=1610, english]{beamer} %If you want to create Polish presentation, then replace 'english' with 'polish'	\usepackage{babel}
\usepackage{babel}
\makeatletter
\@ifclasswith{beamer}{polish}{
	\usepackage{polski}
}
\makeatother
\usepackage[utf8]{inputenc}
\usepackage{listings} % We want to put listings
\usepackage{minted}   % We want to put listings

\mode<beamer>{ 	% In the 'beamer' mode
	\hypersetup{pdfpagemode=FullScreen}         % Enable Full screen mode
	\usetheme[parttitle=rightfooter]{AGH}       % Show part title in right footer
	%\usetheme[nosidebar]{AGH}                  % Do not show sidebar on non-title slides
	%\usetheme[nosidebar, margins=1em]{AGH}     % Do not show sidebar on non-title slides and set both margins (left / right) to 1em
	%\usetheme[dark]{AGH}                       % Use dark background
	%\usetheme[dark, parttitle=leftfooter]{AGH} % Use dark background and show part title in left footer
}
\mode<handout>{	% In the 'handout' mode
	\hypersetup{pdfpagemode=None}		
	\usepackage{pgfpages}
	\pgfpagesuselayout{4 on 1}[a4paper,border shrink=5mm,landscape]	% Show 4 slides on 1 page
	\pgfpageslogicalpageoptions{1}{border code=\pgfusepath{stroke}}
	\pgfpageslogicalpageoptions{2}{border code=\pgfusepath{stroke}}
	\pgfpageslogicalpageoptions{3}{border code=\pgfusepath{stroke}}
	\pgfpageslogicalpageoptions{4}{border code=\pgfusepath{stroke}}
  	\usetheme{boxes}
  	\addheadbox{structure}{\quad\insertpart\hfill\insertsection\hfill\insertsubsection\qquad}          % Content of header
 	\addfootbox{structure}{\quad\insertshortauthor\hfill\insertframenumber\hfill\insertsubtitle\qquad} % Content of footer
}

\AtBeginPart{ % At begin part: display its name
	\frame{\partpage}
} 
\author[Stanisław Polak]{Stanisław Polak\inst{1,2}\newline
   {\scriptsize \href{mailto:polak@agh.edu.pl}{polak@agh.edu.pl}}
}
\date{}
\iflanguage{polish}{
	\title{Temat AGH dla Beamer{\dywiz}a}
	\institute[AGH]{
		\inst{1}Katedra Informatyki\newline
		ul. Kawiory 21\newline
		30-055 Kraków\newline
		\url{http://www.icsr.agh.edu.pl/~polak/}
	\and
		\inst{2}Druga afiliacja
 }
}{
	\title{AGH Beamer Theme}
	\institute[AGH]{
		\inst{1}Institute of Computer Science\newline
		Kawiory 21 Street\newline
		30-055 Kraków\newline
		Poland\newline
		\url{http://www.icsr.agh.edu.pl/~polak/}
	\and
		\inst{2}Second affiliation
}
}
%%%%%%%%%%% Configuration of the listings package %%%%%%%%%%%%%%%%%%%%%%%%%%
% Source: https://en.wikibooks.org/wiki/LaTeX/Source_Code_Listings#Using_the_listings_package
%%%%%%%%%%%%%%%%%%%%%%%%%%%%%%%%%%%%%%%%%%%%%%%%%%%%%%%%%%%%%%%%%%%%%%%%%%%%
\lstset{ %
  backgroundcolor=\color{white},   % choose the background color
  basicstyle=\footnotesize,        % the size of the fonts that are used for the code
  breakatwhitespace=false,         % sets if automatic breaks should only happen at whitespace
  breaklines=true,                 % sets automatic line breaking
  captionpos=b,                    % sets the caption-position to bottom
  commentstyle=\color{green},      % comment style
  deletekeywords={...},            % if you want to delete keywords from the given language
  escapeinside={\%*}{*)},          % if you want to add LaTeX within your code
  extendedchars=true,              % lets you use non-ASCII characters; for 8-bits encodings only, does not work with UTF-8
  frame=single,	                   % adds a frame around the code
  keepspaces=true,                 % keeps spaces in text, useful for keeping indentation of code (possibly needs columns=flexible)
  keywordstyle=\color{blue},       % keyword style
  morekeywords={*,...},            % if you want to add more keywords to the set
  numbers=left,                    % where to put the line-numbers; possible values are (none, left, right)
  numbersep=5pt,                   % how far the line-numbers are from the code
  numberstyle=\tiny\color{gray},   % the style that is used for the line-numbers
  rulecolor=\color{black},         % if not set, the frame-color may be changed on line-breaks within not-black text (e.g. comments (green here))
  showspaces=false,                % show spaces everywhere adding particular underscores; it overrides 'showstringspaces'
  showstringspaces=false,          % underline spaces within strings only
  showtabs=false,                  % show tabs within strings adding particular underscores
  stepnumber=2,                    % the step between two line-numbers. If it's 1, each line will be numbered
  stringstyle=\color{cyan},        % string literal style
  tabsize=2,	                   % sets default tabsize to 2 spaces
  title=\lstname,                  % show the filename of files included with \lstinputlisting; also try caption instead of title
                                   % needed if you want to use UTF-8 Polish chars
  literate={ą}{{\k{a}}}1
           {Ą}{{\k{A}}}1
           {ę}{{\k{e}}}1
           {Ę}{{\k{E}}}1
           {ó}{{\'o}}1
           {Ó}{{\'O}}1
           {ś}{{\'s}}1
           {Ś}{{\'S}}1
           {ł}{{\l{}}}1
           {Ł}{{\L{}}}1
           {ż}{{\.z}}1
           {Ż}{{\.Z}}1
           {ź}{{\'z}}1
           {Ź}{{\'Z}}1
           {ć}{{\'c}}1
           {Ć}{{\'C}}1
           {ń}{{\'n}}1
           {Ń}{{\'N}}1
}
%%%%%%%%%%% Configuration of the minted package %%%%%%%%%%%%%%%%%%%%%%%%%%
\setminted[C++]{frame=single,linenos}

%%%%%%%%%%%%%%%%%
\begin{document}
\maketitle

\ifdefined\textleftmargin
	%%%%%%%%%%%%%%%%
	\begin{frame}[fragile]{\iflanguage{polish}{Informacje}{Information}}
		\begin{center}
			\iflanguage{polish}{
				$\Longleftarrow$ Aktualna wartość \hfill Aktualna wartość $\Longrightarrow$\\
				$\Longleftarrow$ rozmiaru lewego marginesu \hfill rozmiaru prawego marginesu $\Longrightarrow$\\
				$\Longleftarrow$ to \the\textleftmargin \hfill to \the\textrightmargin $\Longrightarrow$
			}{
				$\Longleftarrow$ The current value of \hfill The current value of $\Longrightarrow$\\
				$\Longleftarrow$ the  left  margin size \hfill the right margin size $\Longrightarrow$\\
				$\Longleftarrow$ is \the\textleftmargin \hfill is \the\textrightmargin $\Longrightarrow$
			}
		\end{center}
		\iflanguage{polish}{
			Możesz je zmieniać za pomocą parametru 'margins' \pauza
		}{
			You can change them with the 'margins' parameter ---
		}
		\verb+\usetheme[margins=...]{AGH}+
	\end{frame}
\fi
%%%%%%%%%%%%%%%%
\part{\iflanguage{polish}{Przykłady}{Examples}}
%%%%%%%%%%%%%%%%
\begin{frame}{\iflanguage{polish}{Plan prezentacji}{Outline}}
	\tableofcontents[pausesections]
\end{frame}
%%%%%%%%%%%%%%%%
\section{\iflanguage{polish}{Elementy podstawowe}{Basic elements}}
%%%%%%%%%%%%%%%%
\begin{frame}{\iflanguage{polish}{Wyszczególnienie}{Itemize}}
	\begin{columns}
		\column{0.5\textwidth}
		\begin{itemize}
			\item \iflanguage{polish}{Element 1}{Item 1}
			\item \iflanguage{polish}{Element 2}{Item 2}
			\item \iflanguage{polish}{Element 3}{Item 3}
		\end{itemize}
		\column{0.5\textwidth}
		\pause
		\structure{\iflanguage{polish}{Odkrywanie po kolei}{Uncovering one by one}}
		\begin{itemize}[<+->]
			\item \iflanguage{polish}{Element 1}{Item 1}
			\item \iflanguage{polish}{Element 2}{Item 2}
			\item \iflanguage{polish}{Element 3}{Item 3}
		\end{itemize}
		\onslide
	\end{columns}
\end{frame}
%%%%%%%%%%%%%%%%
\begin{frame}{\iflanguage{polish}{Wyliczenie}{Enumerate}}
	\begin{columns}
		\column{0.5\textwidth}
		\begin{enumerate}
			\item \iflanguage{polish}{Element 1}{Item 1}
			\item \iflanguage{polish}{Element 2}{Item 2}
			\item \iflanguage{polish}{Element 3}{Item 3}
		\end{enumerate}
		\column{0.5\textwidth}
		\pause
		\structure{\iflanguage{polish}{Odkrywanie elementów po kolei z jednoczesnym wyróżnianiem}{Uncovering elements in turn with simultaneous highlighting}}
		\begin{enumerate}[<+-|alert@+>]
			\item \iflanguage{polish}{Element 1}{Item 1}
			\item \iflanguage{polish}{Element 2}{Item 2}
			\item \iflanguage{polish}{Element 3}{Item 3}
		\end{enumerate}
		\onslide
	\end{columns}
\end{frame}
%%%%%%%%%%%%%%%%
\section{\iflanguage{polish}{Matematyka}{Mathematics}}
%%%%%%%%%%%%%%%%
\begin{frame}{\iflanguage{polish}{Podstawowe bloki}{Basic blocks}}
	% Examples from "The beamer class User Guide"
	\iflanguage{polish}{
		\begin{block}{Definicja}
			\alert{Zbiór} składa się z elementów.
		\end{block}
		\begin{exampleblock}{Przykład}
			Zbiór $\{1,2,3,5\}$ zawiera cztery elementy.
		\end{exampleblock}
		\begin{alertblock}{Błędne Twierdzenie}
			$1=2$.
		\end{alertblock}
	}{
		\begin{block}{Definition}
			A \alert{set} consists of elements.
		\end{block}
		\begin{exampleblock}{Example}
			The set $\{1,2,3,5\}$ has four elements.
		\end{exampleblock}
		\begin{alertblock}{Wrong Theorem}
			$1=2$.
		\end{alertblock}
	}
\end{frame}
%%%%%%%%%%%%%%%%
\begin{frame}{\iflanguage{polish}{Otoczenia matematyczne}{Math environments}}
	\begin{columns}
		\column{0.45\textwidth} %The first column
		\structure{\iflanguage{polish}{Twierdzenia}{Theorems}}
		\begin{theorem}[\iflanguage{polish}{Pitagorasa}{Pythagorean}]
			$a^{2}+  b^{2}=  c^{2}$
		\end{theorem}
		\column{0.45\textwidth} %The second column
		\structure{\iflanguage{polish}{Dowody}{Proofs}}
		\begin{proof}
			\ldots
		\end{proof}
	\end{columns}
	\vfill
	\ldots
	\vfill
	\begin{definition}
		\ldots
	\end{definition}
\end{frame}
%%%%%%%%%%%%%%%%
\begin{frame}{\iflanguage{polish}{Dynamiczny wzór matematyczny}{Dynamic mathematical formula}}
	\[
		\binom{n}{k} = \pause \frac{n!}{k!(n-k)!}
	\]
\end{frame}
%%%%%%%%%%%%%%%%
\section{\iflanguage{polish}{Informatyka}{Computer Science}}
%%%%%%%%%%%%%%%%
\subsection*{\iflanguage{polish}{Wstawianie kodów źródłowych}{Inserting source codes}}
%%%%%%%%%%%%%%%%
\begin{frame}[fragile]{\iflanguage{polish}{Użycie otoczenia 'listings'}{Using the 'listings' environment}}
	\begin{lstlisting}[language=C++]
/* The first  program in C++ */  %*\pause*)
#include <iostream>  %*\pause*)
using namespace std; %*\pause*)
void main() 
{       %*\pause*)
  cout %*\pause*) << "Hello World!"%*\pause*) << endl; %*\onslide<4->*)
} %*\onslide*)
\end{lstlisting}
\end{frame}
%%%%%%%%%%%%%%%%
\begin{frame}[fragile]{\iflanguage{polish}{Użycie otoczenia 'minted'}{Using the 'minted' environment}}
	\begin{minted}[beameroverlays,escapeinside=||]{C++}
/* The first  program in C++ */  |\pause|
#include <iostream>  |\pause|
using namespace std; |\pause|
void main() 
{       |\pause|
  cout |\pause| << "Hello World!"|\pause| << endl; |\onslide<4->|
} |\onslide|
	\end{minted}
\end{frame}
%%%%%%%%%%%%%%%%%%%%%%%
\appendix
%%%%%%%%%%%%%%%%%%%%%%%
\begin{frame}[allowframebreaks]{\iflanguage{polish}{Bibliografia}{Bibliography}}
	\begin{thebibliography}{9}
		\setbeamertemplate{bibliography item}[online]
		\bibitem{wikibook}{Wikibooks \newblock \LaTeX/Source Code Listings \newblock \url{https://en.wikibooks.org/wiki/LaTeX/Source_Code_Listings}}
		\bibitem{beamer}{Till Tantau, Joseph Wright, Vedran Miletić \newblock The beamer class \newblock \url{http://mirror.ctan.org/macros/latex/contrib/beamer/doc/beameruserguide.pdf}}
		\setbeamertemplate{bibliography item}[book]
		\bibitem{lamport}{Leslie Lamport \newblock LATEX: a document preparation system : user's guide and reference manual \newblock Addison-Wesley Pub. Co., 1994 }
		\setbeamertemplate{bibliography item}[article]
		\iflanguage{polish}{
			\bibitem{article1}{Autor \newblock Tytuł artykułu \newblock Edytor, rok \newblock Uwagi}
			\setbeamertemplate{bibliography item}[triangle]
			\bibitem{article2}{Autor \newblock Tytuł artykułu \newblock Edytor, rok \newblock Uwagi}
			\setbeamertemplate{bibliography item}[text]
			\bibitem{article3}{Autor \newblock Tytuł artykułu \newblock Edytor, rok \newblock Uwagi}
			\bibitem[Polak98]{article4}{Autor \newblock Tytuł artykułu \newblock Edytor, rok \newblock Uwagi}
		}{
			\bibitem{article1}{Author \newblock Title of the article\newblock Editor, year \newblock Notes}
			\setbeamertemplate{bibliography item}[triangle]
			\bibitem{article2}{Author \newblock Title of the article\newblock Editor, year \newblock Notes}
			\setbeamertemplate{bibliography item}[text]
			\bibitem{article3}{Author \newblock Title of the article\newblock Editor, year \newblock Notes}
			\bibitem[Polak98]{article4}{Author \newblock Title of the article\newblock Editor, year \newblock Notes}
		}
	\end{thebibliography}
\end{frame}
\end{document}
