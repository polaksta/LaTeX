\documentclass[aspectratio=1610,hyperref={pdfpagemode=FullScreen},english]{beamer} %If you want to create Polish presentation, replace 'english' with 'polish'
\usepackage[utf8]{inputenc}
%\usepackage{polski} %For Polish language only
\usepackage{babel}
\usepackage{listings} %We want to put listings

%\usetheme[parttitle=rightfooter]{AGH} %Show part title in right footer
%\usetheme[dark]{AGH}                 %Use dark background
\usetheme[dark,parttitle=leftfooter]{AGH}  %Use dark background and show part title in left footer

\AtBeginPart{\frame{\partpage}} %At begin part: display its name
\title{AGH Beamer Theme}
\author{Stanisław Polak\inst{1,2}}
\date{}
\institute[AGH]{
	\inst{1}Institute of Computer Science\\ul. Kawiory 21\\30-055 Kraków\\
Poland\\
\url{http://www.icsr.agh.edu.pl/~polak/}
\and
  \inst{2}Second affiliation
}
%%%%%%%%%%% Configuration of the listings package %%%%%%%%%%%%%%%%%%%%%%%%%%
% Source: https://en.wikibooks.org/wiki/LaTeX/Source_Code_Listings#Using_the_listings_package
%%%%%%%%%%%%%%%%%%%%%%%%%%%%%%%%%%%%%%%%%%%%%%%%%%%%%%%%%%%%%%%%%%%%%%%%%%%%
\lstset{ %
  backgroundcolor=\color{white},   % choose the background color
  basicstyle=\footnotesize,        % the size of the fonts that are used for the code
  breakatwhitespace=false,         % sets if automatic breaks should only happen at whitespace
  breaklines=true,                 % sets automatic line breaking
  captionpos=b,                    % sets the caption-position to bottom
  commentstyle=\color{green},      % comment style
  deletekeywords={...},            % if you want to delete keywords from the given language
  escapeinside={\%*}{*)},          % if you want to add LaTeX within your code
  extendedchars=true,              % lets you use non-ASCII characters; for 8-bits encodings only, does not work with UTF-8
  frame=single,	                   % adds a frame around the code
  keepspaces=true,                 % keeps spaces in text, useful for keeping indentation of code (possibly needs columns=flexible)
  keywordstyle=\color{blue},       % keyword style
  morekeywords={*,...},            % if you want to add more keywords to the set
  numbers=left,                    % where to put the line-numbers; possible values are (none, left, right)
  numbersep=5pt,                   % how far the line-numbers are from the code
  numberstyle=\tiny\color{gray}, % the style that is used for the line-numbers
  rulecolor=\color{black},         % if not set, the frame-color may be changed on line-breaks within not-black text (e.g. comments (green here))
  showspaces=false,                % show spaces everywhere adding particular underscores; it overrides 'showstringspaces'
  showstringspaces=false,          % underline spaces within strings only
  showtabs=false,                  % show tabs within strings adding particular underscores
  stepnumber=2,                    % the step between two line-numbers. If it's 1, each line will be numbered
  stringstyle=\color{cyan},        % string literal style
  tabsize=2,	                   % sets default tabsize to 2 spaces
  title=\lstname,                  % show the filename of files included with \lstinputlisting; also try caption instead of title
                                   %  needed if you want to use UTF-8 Polish chars
  literate={ą}{{\k{a}}}1
           {Ą}{{\k{A}}}1
           {ę}{{\k{e}}}1
           {Ę}{{\k{E}}}1
           {ó}{{\'o}}1
           {Ó}{{\'O}}1
           {ś}{{\'s}}1
           {Ś}{{\'S}}1
           {ł}{{\l{}}}1
           {Ł}{{\L{}}}1
           {ż}{{\.z}}1
           {Ż}{{\.Z}}1
           {ź}{{\'z}}1
           {Ź}{{\'Z}}1
           {ć}{{\'c}}1
           {Ć}{{\'C}}1
           {ń}{{\'n}}1
           {Ń}{{\'N}}1
}
%%%%%%%%%%%%%%%%%
\begin{document}
  \maketitle
\part{Examples}
%%%%%%%%%%%%%%%%
\begin{frame}{Outline}
  \tableofcontents
\end{frame}
%%%%%%%%%%%%%%%%
\section{Mathematics}
%%%%%%%%%%%%%%%%
\begin{frame}{Basic blocks}
  % Examples from "The beamer class User Guide"
  \begin{block}{Definition}
    A \alert{set} consists of elements.
  \end{block}
  \begin{exampleblock}{Example}
    The set $\{1,2,3,5\}$ has four elements.
  \end{exampleblock}
  \begin{alertblock}{Wrong Theorem}
    $1=2$.
  \end{alertblock}
\end{frame}
%%%%%%%%%%%%%%%%
\begin{frame}{Math environments}
  \begin{columns}
    \column{0.5\textwidth} %The first column
      \structure{Theorems}
      \begin{theorem}[Pythagorean]
        $a^{2}+b^{2}=c^{2}$
      \end{theorem}
      \ldots
    \column{0.5\textwidth} %The second column
      \structure{Proofs}
      \begin{proof}
        \ldots
      \end{proof}
      \ldots
  \end{columns}
\end{frame}
%%%%%%%%%%%%%%%%
\section{Computer Science}
\subsection*{Inserting source codes}
  \begin{frame}[fragile]{Source code}{Appearing step by step}
    \begin{lstlisting}[language=C++]
/* The first  program in C++ */  %*\pause*)
#include <iostream>  %*\pause*)
using namespace std; %*\pause*)
void main() 
{       %*\pause*)
  cout %*\pause*) << "Hello World!"%*\pause*) << endl; %*\onslide<4->*)
} %*\onslide*)
\end{lstlisting}
\end{frame}
\appendix
\begin{frame}{Bibliography}
  \begin{thebibliography}{9}
    \setbeamertemplate{bibliography item}[online]
      \bibitem{wikibook}{Wikibooks \newblock \LaTeX/Source Code Listings \newblock \url{https://en.wikibooks.org/wiki/LaTeX/Source_Code_Listings}}
    \setbeamertemplate{bibliography item}[book]
      \bibitem{lamport}{Leslie Lamport \newblock LATEX: a document preparation system : user's guide and reference manual \newblock Addison-Wesley Pub. Co., 1994 }
    \setbeamertemplate{bibliography item}[article]
      \bibitem{bk1}{Author \newblock Title of the article\newblock Editor, year \newblock Notes}
  \end{thebibliography}
\end{frame}
\end{document}

