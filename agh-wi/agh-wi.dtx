% \iffalse meta-comment
%<*internal>
\iffalse
%</internal>
% %%%%%%%%%%%%%%%%%%%%%%%%%%%%%% README.md %%%%%%%%%%%%%%%%%%%%%%%%%%%%%%%%%%%
%<*readme>
# Klasa `agh-wi`
Klasa LaTeX do prac dyplomowych na Wydziale Informatyki Akademii Górniczo-Hutniczej w Krakowie.

## Opis
Klasa `agh-wi` została stworzona, aby ułatwić studentom Wydziału Informatyki Akademii Górniczo-Hutniczej w Krakowie skład pracy dyplomowej w systemie LaTeX. Rozszerza ona klasę *scrbook*, która wchodzi w skład zestawu klas *KOMA-Script*, oferującego europejskie zasady typografii.

Użytkownik może określić trzy parametry:
1. Język pracy dyplomowej.
2. Nazwa kierunku studiów.
3. Czy praca jest przeznaczona do wyświetlania na ekranie, czy drukowania?

Wartościami domyślnymi, tych parametrów, są, odpowiednio, *język polski*, *Informatyka* oraz *praca będzie oglądana na ekranie*.

Szczegóły użycia klasy można znaleźć w pliku `agh-wi.pdf`.

## Opcje
- **english** — praca dyplomowa jest po angielsku.
     - Teksty wstawiane automatycznie przez LaTeX (spis treści, spis rysunków itp.) są w języku angielskim.
     - Zasady dzielenia wyrazów dotyczą języka angielskiego.
- **data-science** — praca dyplomowa na kierunku „Informatyka — Data Science”.
- **print** — praca dyplomowa będzie drukowana i w związku z tym na każdej stronie powinien być dodany, dodatkowy (2 cm), margines na oprawę.

### Przykłady
1. `\documentclass{agh-wi}` — Treść pracy jest w języku polskim, a na stronie tytułowej jako kierunek studiów należy umieścić napis *Informatyka*. Praca jest przeznaczona do wyświetlania (za pomocą przeglądarki PDF).
2. `\documentclass[english]{agh-wi}` — Treść pracy jest w języku angielskim, a na stronie tytułowej jako kierunek studiów należy umieścić napis *Informatyka*. Praca jest przeznaczona do wyświetlania.
3. `\documentclass[print]{agh-wi}` — Treść pracy jest w języku polskim, na stronie tytułowej jako kierunek studiów należy umieścić napis *Informatyka*. Praca będzie drukowana, a następnie oprawiana.
4. `\documentclass[english, data-science, print]{agh-wi}` — Treść pracy jest w języku angielskim, na stronie tytułowej jako kierunek studiów należy umieścić napis *Informatyka  — Data Science*. Praca będzie drukowana, a następnie oprawiana.

Możesz także używać opcji opisanych w [Podręczniku KOMA-Script](http://mirrors.ctan.org/macros/latex/contrib/koma-script/doc/scrguide-en.pdf)

## Instalacja
Uruchom komendę

```
$ pdflatex agh-wi.dtx
```
do wygenerowania plików `.ins`, `.cls`, `.pdf`, `example.tex` oraz `bibliography.bib`. 

Aby poprawnie wygenerować historię zmian oraz indeks dla dokumentacji klasy, należy uruchomić

```
$ makeindex -s gind.ist agh-wi
$ makeindex -s gglo.ist -o agh-wi.gls agh-wi.glo
$ pdflatex agh-wi.dtx
$ pdflatex agh-wi.dtx
```

(Twoje środowisko TeX może to wykonać za ciebie).

Na koniec, aby użyć wygenerowanego pliku `.cls`, przenieś go do odpowiedniej lokalizacji dla swojej dystrybucji TeX-a. Być może najłatwiejszym sposobem, aby to osiągnąć, jest wykonanie komendy

```
$ mv agh-wi.cls $(kpsewhich -var-value=TEXMFHOME)/tex/latex
```

Jeżeli chcesz udostępnić plik wszystkim użytkownikom urządzenia z systemem Unix, prawdopodobnie lepiej będzie, jak skopiujesz go do katalogu `/usr/local/texlive/texmf-local/tex/latex`, a następnie uruchomisz

```
# texhash
```

Aby mieć pewność, że instalacja się powiodła, wykonaj polecenie

```
$ kpsewhich agh-wi.cls
```
Jeśli wynikiem jest ścieżka do pliku, to znaczy, że poprawnie zainstalowałeś/aś klasę.

## Przykładowa praca dyplomowa
Ze względu na trudności związane z dostarczaniem przykładów użycia komend oraz środowisk w dokumentacji klasy, są one pokazane w oddzielnym pliku — `example.tex` — zawiera on jednocześnie przykładową pracę dyplomową. Możesz go więc potraktować jako szablon do utworzenia własnej pracy.

Aby otrzymać wynikowy dokument PDF, należy skompilować dokument źródłowy za pomocą sekwencji komend:
```
pdflatex -shell-escape example
biber example
makeindex example.nlo  -s nomencl.ist -o example.nls
pdflatex -shell-escape example
pdflatex -shell-escape example
```
%</readme>
% %%%%%%%%%%%%%%%%%%%%%%%%%%%%%%% LICENSE.md %%%%%%%%%%%%%%%%%%%%%%%%%%%%%%%%%%
%<*license>
## agh-wi-wi.dtx
Copyright 2024 Stanisław Polak

This work may be distributed and/or modified under the conditions of the LaTeX
Project Public License, either version 1.3 of this license or (at your option)
any later version.

The latest version of this license is in http://www.latex-project.org/lppl.txt
and version 1.3 or later is part of all distributions of LaTeX version
2005/12/01 or later.

This work has the LPPL maintenance status `maintained'.

The Current Maintainer of this work is Stanisław Polak.
This work consists of the file
* `agh-wi.dtx`
and the derived files
* `agh-wi.ins`,
* `agh-wi.cls`,
* `agh-wi.pdf`,
* `bibliography.bib` and
* `example.tex`.
%</license>
% %%%%%%%%%%%%%%%%%%%%%%%%%%%%%%% example.tex %%%%%%%%%%%%%%%%%%%%%%%%%%%%%%%%%
%<*example>
% W tym przykładzie użyto, między innymi, pakietów:
%   - "minted", co oznacza, że musisz uruchomić kompilator z opcją '-shell-escape'
%   - "biblatex", co oznacza, że po kompilacji (dokumentu) musisz, jeszcze,  wygenerować plik '.bbl'
%   - "nomencl", co oznacza, że za pomocą komendy 'makeindex' musisz wygenerować wykaz symboli
% 
% Tak więc, w celu wygenerowania wynikowego dokumentu PDF, musisz wykonać następujące komendy:
%       pdflatex -shell-escape example
%       biber example
%       makeindex example.nlo  -s nomencl.ist -o example.nls
%       pdflatex -shell-escape example
%       pdflatex -shell-escape example
%
% 
% Dodatkowo musisz mieć zainstalowany program 'pygmentize', który jest częścią pakietu "Pygments" (https://pygments.org/)
% Ww. programy powinny zainstalować się automatycznie podczas instalacji pakietów LaTeX 
%
% Jeżeli masz zainstalowany program 'latexmk', to możesz wygenerować dokument PDF następująco:
%       latexmk example
%       makeindex example.nlo  -s nomencl.ist -o example.nls
%       latexmk example
%
% Do edycji (oraz komplilacji) dokumentów LaTeX polecam program 'TeXstudio' (https://www.texstudio.org/)  
%
% Autor: Stanisław Polak <polak[aT]agh[DoT]edu[DoT]pl>
\documentclass{agh-wi} % Praca po polsku, 
                       % Kierunek 'Informatyka'.
%%%%%%%%%%%%%%%%%%%%%%%%%%%%%%%%%                               
% Inne, przykładowe, użycia klasy
%%%%%%%%%%%%%%%%%%%%%%%%%%%%%%%%%                               
% \documentclass[english]{agh-wi}                % Praca po angielsku,
                                                 % Kierunek 'Informatyka',
                                                 % Przeznaczona do oglądania przy użyciu przegladarki PDF.
% \documentclass[english, data-science]{agh-wi}  % Praca po angielsku,
                                                 % Kierunek 'Informatyka - Data Science',    
                                                 % Przeznaczona do oglądania przy użyciu przegladarki PDF.
% \documentclass[print]{agh-wi}                  % Praca po polsku,
                                                 % Kierunek 'Informatyka',    
                                                 % Przeznaczona do drukowania - każda strona posiada,
                                                 %    dodatkowy (2cm), margines na oprawę.
%%%%%%%%%%%%%%%%%%%%%%%%%%%%%%%%%%%%%
% Parametry dla strony tytułowej
%%%%%%%%%%%%%%%%%%%%%%%%%%%%%%%%%%%%%
\titlePL{Tytuł zgodny z tematyką/dziedziną pracy dyplomowej}  % Tytuł po polsku
\titleEN{Title consistent with the topic/field of the thesis} % Tytuł po angielsku
\author{Imiona i nazwisko dyplomanta}
\supervisor{Stopień lub tytuł naukowy imiona i nazwisko promotora}
%%%%%%%%%%%%%%%%%%%%%%%%%%%%%%%%%%%%%
% Pakiety
%%%%%%%%%%%%%%%%%%%%%%%%%%%%%%%%%%%%%
\usepackage{polski}                           % Obsługa języka polskiego
\usepackage{amsmath}                          % Dodatkowe środowiska matematyczne
\usepackage{amssymb}                          % Dodatkowe symbole matematyczne
\usepackage[polish, intoc]{nomencl}           % Definiowanie symboli
\usepackage{graphicx}                         % Wstawianie grafik zewnętrznych
\usepackage{xcolor}                           % Kolorowy tekst 
\usepackage{tabularx}                         % Rozszerzona wersja środowiska 'tabular'
\usepackage{longtable}                        % Skład "długich" tabel
\usepackage[ruled,linesnumbered]{algorithm2e} % Algorytmy w formie pseudokodu
\usepackage{listings}                         % Wstawianie kodów źródłowych programów
\usepackage[newfloat]{minted}                 % Wstawianie kodów źródłowych programów
\usepackage{caption}                          % Modyfikacji pozycji podpisu
\captionsetup[listing]{position=top}          %       dla listingów 'minted'
\usepackage{hyperref}                         % Obsługa odsyłaczy HTML
\usepackage[                                  % Obsługa spisu literatury
    style=numeric,   % Odnośniki są liczbami
    sorting=none,    % Kolejność pozycji spisu literary <=> kolejności cytowania
    language=autobib,% Zastosuj styl wpisu bibliograficznego 
    autolang=other,  %          właściwy językowi publikacji.
    urldate=iso,     % Zapisuj datę dostępu do strony WWW w formacie RRRR-MM-DD.
    seconds=true,    % Wyświetlaj liczbę sekund w przypadku prezentowania czasów.
    backref=false,   % Nie dodawaj numerów stron, na których występuje cytowanie.
    isbn=true,       % Podawaj ISBN.
    url=false,       % Nie podawaj URL-i, o ile nie jest to konieczne.
    %
    % Ustawienia związane z polskimi normami dla bibliografii.
    maxbibnames=3,
    backend=biber    % Przetwarzaj zawartość blibliograficznej bazy danych za pomocą Bibera
]{biblatex}
%%%%%%%%%%%%%%%%%%%%%%%%%%%%%%%%%%%%
% Ładowanie danych bibliograficznych
%%%%%%%%%%%%%%%%%%%%%%%%%%%%%%%%%%%%
\addbibresource{bibliography.bib}
%%%%%%%%%%%%%%%%%%%%%%%%%%%%%%%%%%%%%
% Opcje konfiguracyjne pakietów
%%%%%%%%%%%%%%%%%%%%%%%%%%%%%%%%%%%%%
% Pakiet 'framed' 
\definecolor{shadecolor}{gray}{0.9}
% Pakiet 'hyperref' 
\hypersetup{
    colorlinks=true,
    linkcolor=blue,
    filecolor=magenta, 
    urlcolor=cyan
} 
% Pakiet 'nomencl'
\makenomenclature % Otwórz plik 'example.nlo'
%%%%%%%%%%%%%%%%%%%%%%%%%%%%%%%%%%%%%
% Definicje komend
%%%%%%%%%%%%%%%%%%%%%%%%%%%%%%%%%%%%%
\newcommand{\alert}[1]{\colorbox{red!50}{#1}}
%%%%%%%%%%%%%%%%%%%%%%%%%%%%%%%%%%%%%
% Twierdzenia i podobne struktury
%%%%%%%%%%%%%%%%%%%%%%%%%%%%%%%%%%%%%
\newtheorem{theorem}{Twierdzenie}
\newtheorem{definition}{Definicja}
%%%%%%%%%%%%%%%%%%%%%%%%%%%%%%%%%%%%%
%%%%%%%%%%%%%%%%%%%%%%%%%%%%%%%%%%%%%
\begin{document}
%%%%%%%%%%%%%%%%%%%%%%%%%%%%%%%%%%%%%
\frontmatter % Część wstępna
%%%%%%%%%%%%%%%%%%%%%%%%%%%%%%%%%%%%%
\maketitle % Dodaj stronę tytułową
%%%%%%%%%%%%%%%%%%%%%%%
% Jeżeli chcesz komuś podziękować, to możesz użyć poniższego kodu
\cleardoublepage
\thispagestyle{empty}
\vspace*{\fill}
\begin{flushright}
    \em
    \begin{minipage}{0.75\textwidth}
        Tutaj możesz umieścić treść podziękowań.
        Tutaj możesz umieścić treść podziękowań.
        Tutaj możesz umieścić treść podziękowań.
        Tutaj możesz umieścić treść podziękowań.
        Tutaj możesz umieścić treść podziękowań.
    \end{minipage}
\end{flushright}
%%%%%%%%%%%%%%%%%%%%%%%
\begin{abstractPL}
    Streszczenie po polsku \ldots
\end{abstractPL}
\begin{abstractEN}
    Abstract in english \ldots
\end{abstractEN}
%%%%%%%%%%%%%%%%%%%%%%%%%%%%%%%%%%%%%
%%%%%%%%%%%%%%%%%%%%%%%%%%%%%%%%%%%%%
\tableofcontents   % Wygeneruj spis treści
\begin{shaded}
    Zawartość spisu treści \pauza tytuły rozdziałów oraz ich ilość zależą od tematyki pracy \pauza należy ustalić z opiekunem pracy.
\end{shaded}
\listoffigures     % Wygeneruj listę rysunków
\listoftables      % Wygeneruj listę tabel
\listofalgorithms  % Wygeneruj listę algorytmów
% Wygeneruj listę kodów źrółowych
\lstlistoflistings % Jeżeli do tworzenia listingów używasz pakietu 'listings'
% \listoflistings  % Jeżeli do tworzenia listingów używasz pakietu 'minted'
\printnomenclature % Wyświetl listę symboli
%%%%%%%%%%%%%%%%%%%%%%%%%%%%%%%%%%%%%
\mainmatter % Część główna
%%%%%%%%%%%%%%%%%%%%%%%%%%%%%%%%%%%%%
\chapter{Wstęp}
\begin{shaded}
    Tytuł oraz strukturę rozdziału należy ustalić z opiekunem pracy.
\end{shaded}
Wprowadzenie w tematykę pracy.
\section{Cel i zakres pracy}
Streszczenie specyfikacji wymagań Promotora.
\chapter{Część literaturowa}
\begin{shaded}
    Tytuł oraz strukturę rozdziału należy ustalić z opiekunem pracy.
\end{shaded}

Aktualny stan wiedzy, na dany temat, na podstawie dostępnej literatury naukowej oraz specjalistycznej.
\chapter{Część badawcza}
\begin{shaded}
    Tytuł oraz strukturę rozdziału należy ustalić z opiekunem pracy.
\end{shaded}

\begin{itemize}
    \item Problemy / pytania badawcze.
    \item Opis idei / metod rozwiązania postawionego problemu.
    \item Opis przebiegu badań.
    \item Interpretacja uzyskanych wyników.
\end{itemize}
\chapter{Zakończenie}
\begin{shaded}
    Tytuł oraz strukturę rozdziału należy ustalić z opiekunem pracy.
\end{shaded}
\begin{enumerate}
    \item Podsumowanie.
    \item Możliwości dalszego rozwoju.
    \item Potencjalne obszary zastosowania pracy.
\end{enumerate}
%%%%%%%%%%%%%%%%%%%%%%%%%%%%%%%%%%%%%
%%%%%%%%%%%%%%%%%%%%%%%%%%%%%%%%%%%%%
\appendix % Dodatek
\chapter{Typowe elementy składowe pracy dyplomowej z Informatyki}
\section{Tabele}
W tabeli \ref{tab:result} przedstawiono wyniki pomiarów.
\begin{shaded}
    Podpis ma być przed tabelą.
\end{shaded}
\begin{table}[!h]
    \caption{Pomiary zużycia energii elektrycznej\label{tab:result}.}
    \centering
    \begin{tabular}{|l||r@{,}l|}
        \hline
        \textbf{L.p.} & \multicolumn{2}{|c|}{\textbf{Wartość}}          \\
        \hline
        \hline
        \cline{2-3}
        %\textbf{L.p.} & \multicolumn{1}{r@{\,\vline\,}}{Całkowita} & Ułamkowa \\
        \hline
        1             & 12345                                  & 6789   \\
        \cline{2-3}
                      & 45                                     & 89     \\
        \hline
        2             & 45                                     & 678901 \\
        \hline
    \end{tabular}
\end{table}

Jeżeli tabela zawiera dużą liczbę wierszy i może nie zmieścić się na stronie \pauza patrz tabela \ref{tab:longtable} \pauza należy skorzystać z pakietu \emph{longtable} \cite{longtable}.
\begin{longtable}{|p{14ex}|p{1.5em}|p{1.5em}|p{1.5em}|p{1.5em}|p{1.5em}|p{1.5em}|p{1.5em}|p{1.5em}|p{1.5em}|}
    \caption{Tabela, która zawiera dużą ilość wierszy\label{tab:longtable}.}
    \endfirsthead
    \hline
               & 1 & 2 & 3 & 4 & 5 & 6 & 7 & 8 & \\
    \cline{2-9}
               &   &   &   &   &   &   &   &   & \\
    \hline
    \endhead
    \endfoot
    \hline\hline
               & 1 & 2 & 3 & 4 & 5 & 6 & 7 & 8 & \\
    \cline{2-9}
               &   &   &   &   &   &   &   &   & \\
    \hline\hline
    Student 1  &   &   &   &   &   &   &   &   & \\
    \cline{2-9}
               &   &   &   &   &   &   &   &   & \\
    \hline\hline
    Student 2  &   &   &   &   &   &   &   &   & \\
    \cline{2-9}
               &   &   &   &   &   &   &   &   & \\
    \hline\hline
    Student 3  &   &   &   &   &   &   &   &   & \\
    \cline{2-9}
               &   &   &   &   &   &   &   &   & \\
    \hline\hline
    Student 4  &   &   &   &   &   &   &   &   & \\
    \cline{2-9}
               &   &   &   &   &   &   &   &   & \\
    \hline\hline
    Student 5  &   &   &   &   &   &   &   &   & \\
    \cline{2-9}
               &   &   &   &   &   &   &   &   & \\
    \hline\hline
    Student 6  &   &   &   &   &   &   &   &   & \\
    \cline{2-9}
               &   &   &   &   &   &   &   &   & \\
    \hline\hline
    Student 7  &   &   &   &   &   &   &   &   & \\
    \cline{2-9}
               &   &   &   &   &   &   &   &   & \\
    \hline\hline
    Student 8  &   &   &   &   &   &   &   &   & \\
    \cline{2-9}
               &   &   &   &   &   &   &   &   & \\
    \hline\hline
    Student 9  &   &   &   &   &   &   &   &   & \\
    \cline{2-9}
               &   &   &   &   &   &   &   &   & \\
    \hline\hline
    Student 10 &   &   &   &   &   &   &   &   & \\
    \cline{2-9}
               &   &   &   &   &   &   &   &   & \\
    \hline\hline
    Student 11 &   &   &   &   &   &   &   &   & \\
    \cline{2-9}
               &   &   &   &   &   &   &   &   & \\
    \hline\hline
    Student 12 &   &   &   &   &   &   &   &   & \\
    \cline{2-9}
               &   &   &   &   &   &   &   &   & \\
    \hline\hline
    Student 13 &   &   &   &   &   &   &   &   & \\
    \cline{2-9}
               &   &   &   &   &   &   &   &   & \\
    \hline\hline
\end{longtable}

Tabele, w których występuje długi tekst, a co za tym idzie może się on nie zmieścić \pauza musi zostać zawinięty, należy składać za pomocą środowiska 'tabularx' \cite{tabularx} zamiast 'tabular' \pauza  patrz tabela \ref{tab:tabularx}.
\begin{table}[!h]
    \caption{Tabela zawierająca długi tekst\label{tab:tabularx}.}
    \centering
    \begin{tabularx}{300pt}{|c|X|c|X|}
        \hline
        \multicolumn{2}{|c|}{Wpis wielokolumnowy!} &
        TRZY                                       &
        CZTERY                                       \\
        \hline
        jeden                                      &
        \raggedright\arraybackslash Szerokość tej kolumny zależy od
        szerokości tabeli.                         &
        trzy                                       &
        \raggedright\arraybackslash Kolumna czwarta będzie zachowywać się w taki sam sposób jak
        druga kolumna o tej samej szerokości.        \\
        \hline
    \end{tabularx}
\end{table}

\paragraph{Uwaga}
\alert{Każda tabela powinna być opisana w treści pracy.}
%%%%%%%%%%%%%%%%%%%%%%%%%%%%%%%%%%%%%
\section{Rysunki}
\paragraph{Uwagi}
\begin{itemize}
    \item Rysunki powinny być przerysowane samodzielnie albo używane tylko te,
          których twórcy zezwolili na ich rozpowszechnianie oraz kopiowanie, czyli
          np. rysunki objęte licencją Creative Commons.
    \item \alert{Każdy rysunek powinien być opisany w treści pracy.}
\end{itemize}
\subsection{Wewnętrzne}

Klasa \emph{agh-wi}, automatycznie, dołącza pakiet \emph{TikZ} \cite{tikz} \pauza dostarcza on komend pozwalających na tworzenie grafik. Przykładowe grafiki pokazano na rysunku \ref{fig:tikz1} oraz \ref{fig:tikz2}.

\begin{figure}[!h]
    \begin{center}
        \tikz \draw[thick,rounded corners=8pt]
        (0,0) -- (0,2) -- (1,3.25) -- (2,2) -- (2,0) -- (0,2) -- (2,2) -- (0,0) -- (2,0);
    \end{center}
    \caption{Prosty rysunek \emph{TikZ}\label{fig:tikz1}.}
\end{figure}

\begin{figure}[!h]
    \begin{center}
        \begin{tikzpicture}
            \draw[step=.5cm,gray,very thin] (-1.4,-1.4) grid (1.4,1.4);
            \draw (-1.5,0) -- (1.5,0);
            \draw (0,-1.5) -- (0,1.5);
            \draw (0,0) circle [radius=1cm];
        \end{tikzpicture}
    \end{center}
    \caption{Bardziej złożony rysunek \emph{TikZ}\label{fig:tikz2}.}
\end{figure}

\subsection{Zewnętrzne}
Oczywiście możliwe jest również dołączanie rysunków zewnętrznych \pauza
pakiet \emph{graphicx} \cite{graphicx} pozwala na wstawianie grafik zapisanych w  plikach: '.png', '.jpg' oraz '.pdf'. Rysunek \ref{fig:logo} \pauza patrz strona \pageref{fig:logo} \pauza wstawiono przy użyciu tego pakietu.
\begin{figure}[!h]
    \begin{center}
        \IfFileExists{logo_podstawowe.png}{
            \includegraphics[width=0.7\linewidth]{logo_podstawowe}
        }
        {Nie znaleziono pliku 'logo\_podstawowe.png' \pauza pobierz go ze strony \url{https://www.informatyka.agh.edu.pl/media/uploads/Logo WI/PNG/logo_podstawowe.png}}
    \end{center}
    \caption{Logo Wydziału Informatyki.}
    \label{fig:logo}
\end{figure}
%%%%%%%%%%%%%%%%%%%%%%%%%%%%%%%%%%%%%
\section{Algorytmy}
Pakiet \emph{algorithm2e} \cite{algorithm2e} pozwala zapisywać algorytmy w formie pseudokodu \pauza patrz
algorytm \ref{alg:algo_disjdecomp} na stronie \pageref{alg:algo_disjdecomp}.
\begin{shaded}
    Podpis ma być przed algorytmem.
\end{shaded}
\begin{algorithm}[!htb]
    \SetKwData{Left}{left}\SetKwData{This}{this}\SetKwData{Up}{up}
    \SetKwFunction{Union}{Union}\SetKwFunction{FindCompress}{FindCompress}
    \SetKwInOut{Input}{input}\SetKwInOut{Output}{output}
    \Input{A bitmap $Im$ of size $w\times l$}
    \Output{A partition of the bitmap}
    \BlankLine
    \emph{special treatment of the first line}\;
    \For{$i\leftarrow 2$ \KwTo $l$}{
        \emph{special treatment of the first element of line $i$}\;
        \For{$j\leftarrow 2$ \KwTo $w$}{\label{forins}
            \Left$\leftarrow$ \FindCompress{$Im[i,j-1]$}\;
            \Up$\leftarrow$ \FindCompress{$Im[i-1,]$}\;
            \This$\leftarrow$ \FindCompress{$Im[i,j]$}\;
            \If(\tcp*[h]{O(\Left,\This)==1}){\Left compatible with \This}{\label{lt}
                \lIf{\Left $<$ \This}{\Union{\Left,\This}}
                \lElse{\Union{\This,\Left}}
            }
            \If(\tcp*[f]{O(\Up,\This)==1}){\Up compatible with \This}{\label{ut}
                \lIf{\Up $<$ \This}{\Union{\Up,\This}}
                \tcp{\This is put under \Up to keep tree as flat as possible}\label{cmt}
                \lElse{\Union{\This,\Up}}\tcp*[h]{\This linked to \Up}\label{lelse}
            }
        }
        \lForEach{element $e$ of the line $i$}{\FindCompress{p}}
    }
    \caption{disjoint decomposition.}\label{alg:algo_disjdecomp}
\end{algorithm}\DecMargin{1em}

%%%%%%%%%%%%%%%%%%%%%%%%%%%%%%%%%%%%%
\section{Kody źródłowe}
Najpopularniejszymi pakietami, które umożliwiają składanie kodów źródłowych programów, są:
\begin{description}
    \item[listings \cite{listings}] \pauza kod źródłowy jest formatowany bezpośrednio przez \LaTeX{}\dywiz{}a \pauza nie jest używany żaden, zewnętrzny, formater kodu.
        \begin{lstlisting}[language=C++, float=ht, label=lst:code1, caption={Przykładowy kod źródłowy sformatowany za pomocą pakietu 'listings'.}]
/* Pierwszy program w C++ */
        
#include <iostream>
        
int main() {
    std::cout << "Hello World!";
    return 0;
}
    \end{lstlisting}
    \item[minted \cite{minted}] \pauza formatuje kod źródłowy przy użyciu biblioteki języka Python  o nazwie \emph{Pygments} \cite{pygments}.
        \begin{listing}[!ht]
            \caption{Przykładowy listing sformatowany za pomocą pakietu 'minted'.\label{lst:code2}}
            \begin{minted}{c++}
/* Pierwszy program w C++ */

#include <iostream>

int main() {
    std::cout << "Hello World!";
    return 0;
}        
    \end{minted}
        \end{listing}
\end{description}

Kod źródłowy w C++ sformatowany przy użyciu pakietu \emph{listings}, pokazano na listingu \ref{lst:code1}; sformatowany przy użyciu pakietu \emph{minted}, pokazano na listingu \ref{lst:code2}.
\begin{shaded}
    \begin{itemize}
        \item Podpis ma być przed kodem źródłowym.
        \item \alert{Proszę używać tylko jednego z tych pakietów.} W przeciwnym razie otrzymasz taki efekt, jak w przykładowej pracy \pauza obydwa listingi mają ten sam numer.
    \end{itemize}
\end{shaded}
%%%%%%%%%%%%%%%%%%%%%%%%%%%%%%%%%%%%%
\section{Wzory}
\begin{shaded}
    Należy używać tylko dwóch rodzajów wzorów:
    \begin{enumerate}
        \item ,,W linii''.
        \item Eksponowane, numerowane.
    \end{enumerate}
\end{shaded}
\LaTeX{} bardzo dobrze sprawdza się w przypadku prac dyplomowych zawierających wzory matematyczne\footnote{ W przypadku złożonych wzorów warto zastosować pakiet \emph{amsmath} \cite{amsmath}.}.
\subsection{Przykłady}
Wzór $E = mc^{2}$\nomenclature{$c$}{Prędkość światła w próżni} jest częścią zdania.

\begin{equation}
    \left|\sum_{i=1}^n a_ib_i\right|
    \le
    \left(\sum_{i=1}^n a_i^2\right)^{1/2}
    \left(\sum_{i=1}^n b_i^2\right)^{1/2}
\end{equation}

% Do wstawienia poniższych wzorów użyto środowisk (otoczeń) zdefiniowanych w pakiecie 'amsmath'
Wartości zmiennej opisano wzorem \ref{eq:r1}.
\begin{equation}
    \label{eq:r1}
    x=\begin{cases}
        y           & \text{dla } y > 0    \\
        \frac{z}{y} & \text{dla } y \leq 0
    \end{cases}
\end{equation}

Wzór \ref{eq:r2} to wzór wielowierszowy.
\begin{align}
    \label{eq:r2}
    2x^2 + 3(x-1)(x-2) & =2x^2 + 3(x^2-3x+2)             \\
                       & = 2x^2 + 3x^2 - 9x + 6\nonumber \\
                       & = 5x^2 - 9x + 6\nonumber
\end{align}
%%%%%%%%%%%%%%%%%%%%%%%%%%%%%%%%%%%%%
\section{Twierdzenia i podobne struktury}

Twierdzenie nr \ref{tw} opublikował, w roku 1691, francuski matematyk Michel Rolle.
\begin{theorem}[Rolle'a]
    \label{tw}
    Jeśli dana funkcja f: $\mathbb R \to \mathbb R$ jest:
    \begin{enumerate}
        \item ciągła w przedziale $[a,b]$
        \item jest różniczkowalna w przedziale $(a,b)$
        \item na końcach przedziału $[a,b]$ przyjmuje równe wartości: $f(a) = f(b)$,
    \end{enumerate}
    to w przedziale $(a,b)$ istnieje co najmniej jeden punkt c taki, że $f'(c) = 0$.
\end{theorem}


Teraz coś z informatyki \ldots
\begin{definition}
    Bit to najmniejsza jednostka informacji w komputerze.
\end{definition}
\begin{definition}
    Bajtem nazywamy ciąg ośmiu bitów.
\end{definition}
%%%%%%%%%%%%%%%%%%%%%%%%%%%%%%%%%%%%%
\backmatter % Część końcowa
%%%%%%%%%%%%%%%%%%%%%%%%%%%%%%%%%%%%%
\chapter{Uwagi Autora}
\begin{itemize}
    \item Aktualna wersja klasy jest dostępna pod adresem \url{https://github.com/polaksta/LaTeX/tree/master/agh-wi}.
    \item Skoro Twoja praca dyplomowa powstała w \LaTeX{}u, to zachęcam Cię również do przygotowania prezentacji (na obronę pracy magisterskiej) w tym języku. Najpopularniejszą klasą do tworzenia tego typu dokumentów jest \emph{beamer} \cite{beamer}.
    \item Pod adresem \url{https://github.com/polaksta/LaTeX/tree/master/beamerthemeAGH} możesz znaleźć szablon 'beamer' mojego autorstwa.
    \item Treść wszystkich rozdziałów tej, przykładowej, pracy dyplomowej znajduje się w jednym pliku \pauza \textbf{nie jest to polecane rozwiązanie}. W przypadku pisania własnej pracy warto umieścić zawartość każdego z rozdziałów w osobnych plikach, a następnie dołączać je do dokumentu głównego \pauza patrz opis na stronie \url{https://www.dickimaw-books.com/latex/thesis/html/include.html}.
    \item Jeżeli pewne elementy mają być wyróżniane w \alert{jednakowy} \alert{sposób}, to proponuję nie używać bezpośredniego stylowania, tzn.
          \mint{tex}{\colorbox{red!50}{jednakowy} \colorbox{red!50}{sposób}} ale zdefiniować własną komendę stylującą, np. \verb+\alert+,
          \mint{tex}{\newcommand{\alert}[1]{\colorbox{red!50}{#1}}}
          a następnie użyć jej w dokumencie.
          \mint{tex}{\alert{jednakowy} \alert{sposób}}

          Dzięki temu, jeżeli będziesz chciał / chciała zmienić sposób stylowania tych elementów, np. niebieskie tło zamiast czerwonego, to wystarczy zmodyfikować, tylko, definicję komendy, zamiast zastępować, w tekście pracy dyplomowej, wybrane (niekoniecznie wszystkie!) wystąpienia tekstu \texttt{red}, tekstem \texttt{blue}.
\end{itemize}
Stanisław Polak

%%%%%%%%%%%%%%%%%%%%%%%%%%%%%%%%%%%%%
%%%%%%%%%%%%%%%%%%%%%%%%%%%%%%%%%%%%%
% Wyświetl spis literatury
% UWAGA: Powinien zawierać tylko te publikacje, do których odwołujesz się w pracy 
\printbibliography
%%%%%%%%%%%%%%%%%%%%%%%%%%%%%%%%%%%%%
%%%%%%%%%%%%%%%%%%%%%%%%%%%%%%%%%%%%%
\end{document}
%</example>
% %%%%%%%%%%%%%%%%%%%%%%%%%%%%%%% bibliography.bib %%%%%%%%%%%%%%%%%%%%%%%%%%%%%%%%%
%<*bibliography>
@online{algorithm2e,
	title = {algorithm2e.sty — package for algorithms},
	url = {http://mirrors.ctan.org/macros/latex/contrib/algorithm2e/doc/algorithm2e.pdf},
}

@online{listings,
	title = {The Listings Package},
	url = {http://mirrors.ctan.org/macros/latex/contrib/listings/listings.pdf},
}

@online{minted,
	title = {The minted package: Highlighted source code in \LaTeX},
	url = {http://mirrors.ctan.org/macros/latex/contrib/minted/minted.pdf},
}

@online{amsmath,
	title = {User’s Guide for the amsmath Package},
	url = {http://mirrors.ctan.org/macros/latex/required/amsmath/amsldoc.pdf},
}

@online{graphicx,
	title = {Packages in the ‘graphics’ bundle},
	url = {http://mirrors.ctan.org/macros/latex/required/graphics/grfguide.pdf},
}

@online{tikz,
	title = {The TikZ and PGF Packages},
	url = {http://mirrors.ctan.org/graphics/pgf/base/doc/pgfmanual.pdf},
}

@online{beamer,
	title = {The beamer class},
	url = {http://mirrors.ctan.org/macros/latex/contrib/beamer/doc/beameruserguide.pdf},
}

@online{tabularx,
	title = {The tabularx package},
	url = {http://mirrors.ctan.org/macros/latex/required/tools/tabularx.pdf},
}

@online{longtable,
	title = {The longtable package},
	url = {http://mirrors.ctan.org/macros/latex/required/tools/longtable.pdf},
}

@online{pygments,
	title = {Strona WWW biblioteki 'Pygments'},
	url = {https://pygments.org/},
}
%</bibliography>
%%%%%%%%%%%%%%%%%%%%%%%%%%%%%%%%%%%%%%%%%%%%%%%%%%%%%%%%%%%%%%%%%%%%%%%%%%%%%
%<*internal>
\fi
\def\nameofplainTeX{plain}
\ifx\fmtname\nameofplainTeX\else
  \expandafter\begingroup
\fi
%</internal>
%<*install>
\input docstrip.tex
\keepsilent
\askforoverwritefalse
    \preamble
----------------------------------------------------------------
agh-wi --- Diploma thesis template for students of the Faculty of Computer Science of the AGH University of Krakow
email: polak[aT]agh[DoT]edu[DoT]pl
GitHub: https://github.com/polaksta
Released under the LaTeX Project Public License v1.3c or later
See http://www.latex-project.org/lppl.txt
----------------------------------------------------------------

\endpreamble
    \postamble

Copyright (C) 2024 by Stanisław Polak <polak[aT]agh[DoT]edu[DoT]pl>

This work may be distributed and/or modified under the conditions of the LaTeX
Project Public License (LPPL), either version 1.3c of this license or (at your
option) any later version. The latest version of this license is in the file:

http://www.latex-project.org/lppl.txt

This work is "maintained" (as per LPPL maintenance status) by
Stanisław Polak.
This work consists of the file
agh-wi.dtx
and the derived files
agh-wi.ins,
agh-wi.cls,
agh-wi.pdf,
bibliography.bib, and
example.tex.

\endpostamble
\usedir{tex/latex/agh}
\generate{
  \file{example.tex}{\from{\jobname.dtx}{example}}
}
\usedir{tex/latex/agh}
\generate{
  \file{\jobname.cls}{\from{\jobname.dtx}{class}}
}
\usedir{tex/latex/agh}
\generate{
  \file{bibliography.bib}{\from{\jobname.dtx}{bibliography}}
}
%</install>
%<install>\endbatchfile
%<*internal>
\usedir{source/latex/agh}
\generate{
  \file{\jobname.ins}{\from{\jobname.dtx}{install}}
}
\nopreamble\nopostamble
\usedir{doc/latex/agh}
\generate{
  \file{README.md}{\from{\jobname.dtx}{readme}}
}
\generate{
  \file{LICENSE.md}{\from{\jobname.dtx}{license}}
}
\ifx\fmtname\nameofplainTeX
  \expandafter\endbatchfile
\else
  \expandafter\endgroup
\fi
%</internal>
%<*class>
\def\fileversion{1.0}
\def\filedate{2024/02/14}
\NeedsTeXFormat{LaTeX2e}
\ProvidesClass{agh-wi}[\filedate\space\fileversion\space Diploma thesis at WI AGH Kraków, Poland]
%</class>
%<*driver>
\documentclass{ltxdoc}
\usepackage[T1]{fontenc} % you should always use this, even with English
                         % see http://tex.stackexchange.com/a/677/51283
\usepackage{lmodern}
\usepackage[numbered]{hypdoc}
\EnableCrossrefs
\CodelineIndex
\RecordChanges
\begin{document}
\newcommand{\mode}{}
\newcommand{\ifAGH}{}
\newcommand{\colorlet}{}
\newcommand{\AGH}{}
\newcommand{\tikzstyle}{}
  \DocInput{\jobname.dtx}
\end{document}
%</driver>
% \fi
%\title{^^A
%  \textsf{agh-wi} --- Diploma thesis at \emph{WI AGH}\thanks{^^A
%    This file describes version \fileversion, last revised \filedate\space ^^A
%  }^^A
%}
%\author{^^A
%  Stanisław Polak\thanks{email: polak[aT]agh[DoT]edu[DoT]pl}^^A
%}
%\date{Released \filedate}
%
% %%%%%%%%%%%%%%%%%%%%%%%%%%%%%%%%%%%%%%%%%%%%%%%%%%%%%%%%%%%%%%%%%%%%%%%%%%%%%
%
%\DoNotIndex{\LoadClass}
%\DoNotIndex{\newcommand}
%\DoNotIndex{\newenvironment}
%\DoNotIndex{\null}
%\DoNotIndex{\typeout}
%\DoNotIndex{\newif}
%\DoNotIndex{\def}
%\DoNotIndex{\ifenglish}
%\DoNotIndex{\ifprint}
%\DoNotIndex{\logoscale}
%\DoNotIndex{\fieldOfStudy}
%\DoNotIndex{\changes}
%\DoNotIndex{\DeclareOption}
%\DoNotIndex{\englishtrue}
%\DoNotIndex{\printtrue}
%\DoNotIndex{\PassOptionsToClass}
%\DoNotIndex{\CurrentOption}
%\DoNotIndex{\ProcessOptions}
%\DoNotIndex{\RequirePackage}
%\DoNotIndex{\KOMAoptions}
%\DoNotIndex{\baselineskip}
%\DoNotIndex{\begin}
%\DoNotIndex{\bfseries}
%\DoNotIndex{\centering}
%\DoNotIndex{\definecolor}
%\DoNotIndex{\else}
%\DoNotIndex{\end}
%\DoNotIndex{\endtitlepage}
%\DoNotIndex{\fi}
%\DoNotIndex{\gdef}
%\DoNotIndex{\Large}
%\DoNotIndex{\large}
%\DoNotIndex{\par}
%\DoNotIndex{\path}
%\DoNotIndex{\relax}
%\DoNotIndex{\renewcommand}
%\DoNotIndex{\scshape}
%\DoNotIndex{\sffamily}
%\DoNotIndex{\small}
%\DoNotIndex{\textwidth}
%\DoNotIndex{\titlepage}
%\DoNotIndex{\the}
%\DoNotIndex{\tikz}
%\DoNotIndex{\tikzset}
%\DoNotIndex{\vfil}
%\DoNotIndex{\vfill}
%\DoNotIndex{\vspace}
%\DoNotIndex{\space}
%\DoNotIndex{\year}
%\DoNotIndex{\addtocontents}
%\DoNotIndex{\algorithmcfname}
%\DoNotIndex{\appendixmore}
%\DoNotIndex{\appendixname}
%\DoNotIndex{\appendixprefixformat}
%\DoNotIndex{\AtBeginDocument}
%\DoNotIndex{\color}
%\DoNotIndex{\csname}
%\DoNotIndex{\DeclareTOCStyleEntry}
%\DoNotIndex{\endcsname}
%\DoNotIndex{\footnotesize}
%\DoNotIndex{\k}
%\DoNotIndex{\l}
%\DoNotIndex{\L}
%\DoNotIndex{\listalgorithmcfname}
%\DoNotIndex{\lstlistingname}
%\DoNotIndex{\lstlistlistingname}
%\DoNotIndex{\lstset}
%\DoNotIndex{\setminted}
%\DoNotIndex{\SetupFloatingEnvironment}
%\DoNotIndex{\string}
%\DoNotIndex{\tiny}
%\DoNotIndex{\tocchapapp}
%\DoNotIndex{\ttfamily}
%\DoNotIndex{\\}
%\DoNotIndex{\%}
%\DoNotIndex{\'}
%\DoNotIndex{\.}
%\DoNotIndex{\@M}
%\DoNotIndex{\@author}
%\DoNotIndex{\@beginparpenalty}
%\DoNotIndex{\@endparpenalty}
%\DoNotIndex{\@endparpenalty}
%\DoNotIndex{\@lowpenalty}
%\DoNotIndex{\@supervisor}
%\DoNotIndex{\@titleEN}
%\DoNotIndex{\@titlePL}
%\DoNotIndex{\@ifpackageloaded}

%
% %%%%%%%%%%%%%%%%%%%%%%%%%%%%%%%%%%%%%%%%%%%%%%%%%%%%%%%%%%%%%%%%%%%%%%%%%%%%%
%
%\maketitle
%
%\changes{v1.0}{2024/02/08}{initial release}
%
%\begin{abstract}
%  \LaTeX{} class for diploma theses at the Faculty of Computer Science of the AGH University of Krakow, Poland.
%\end{abstract}
%
%\section{Documentation}
% The class defines the following commands and environments:
%
%\DescribeMacro{\supervisor}
% Name and surname of the diploma thesis supervisor.
% 
%\DescribeMacro{\titlePL}
% Polish title of the diploma thesis.
%
%\DescribeMacro{\titleEN}
% English title of the diploma thesis.
%
%\DescribeMacro{\maketitle}
% Generating the title page of the diploma thesis.
%
%\DescribeEnv{abstractPL}
% Polish-language abstract of the diploma thesis.
% 
%\DescribeEnv{abstractEN}
% English-language abstract of the diploma thesis.
%
%\StopEventually{^^A
%  \PrintChanges
%  \PrintIndex
%}
%
%\section{Default values}
%
%    \begin{macrocode}
%<*class>
%    \end{macrocode}
\typeout{..................................................................}
\typeout{You are using the 'agh-wi' class created by Stanisław Polak}
\typeout{Version: \fileversion, \filedate}
\typeout{..................................................................}
% By default:
%
% The typesetting is performed for the Polish language.
%    \begin{macrocode}
\newif\ifenglish
%    \end{macrocode}
%
% Version for viewing, using a PDF viewer, is created.
%    \begin{macrocode}
\newif\ifprint
%    \end{macrocode}
%    
% The field of study --- visible on the title page --- is 'Computer Science'.
%    \begin{macrocode}
\def\fieldOfStudy{Informatyka}
%    \end{macrocode}
% The abovementioned values can be changed using options.
% \section{Options}
%
% The typesetting is performed for the English language.
%    \begin{macrocode}
\DeclareOption{english}{\englishtrue}
%    \end{macrocode}
% Create a version intended for printing.
%    \begin{macrocode}
\DeclareOption{print}{\printtrue}
%    \end{macrocode}
% The field of study is 'Computer Science --- Data Science'.
%    \begin{macrocode}
\DeclareOption{data-science}{\def\fieldOfStudy{Informatyka --- Data Science}}
%    \end{macrocode}
% If the option is not matched, pass it to the 'scrbook' class.
%    \begin{macrocode}
\DeclareOption*{\PassOptionsToClass{\CurrentOption}{scrbook}}
%    \end{macrocode}
% \section{Option Processing}
%    \begin{macrocode}
\ProcessOptions\relax
%    \end{macrocode}
% \section{Loading scrbook}
% The class inherits from the 'scrbook' class.
%    \begin{macrocode}
\LoadClass[ appendixprefix=true]{scrbook}
%    \end{macrocode}
%
% \section{Required packages}
% The image of the AGH logo on the title page is drawn using the commands of the \emph{tikz} package.
%    \begin{macrocode}
\RequirePackage{tikz}
%    \end{macrocode}
%
%  The \emph{scrlayer-scrpage} package allows the user to define and manage page styles by controlling page headers and footers.
%    \begin{macrocode}
\RequirePackage{scrlayer-scrpage}
\ifprint
    \KOMAoptions{fontsize=11pt, DIV=14, BCOR=2cm, automark, headsepline, footsepline, plainfootsepline,toc=listof, toc=bib}
\else
    \KOMAoptions{fontsize=11pt, DIV=14, automark, headsepline, footsepline, plainfootsepline, toc=listof, toc=bib}
\fi
%    \end{macrocode}
%
% The \emph{babel} package allows you to typeset document text in a specific language.
%    \begin{macrocode}
\ifenglish
    \RequirePackage[main=english,polish]{babel}
\else
    \RequirePackage[main=polish,english]{babel}
\fi
%    \end{macrocode}
% 
% The first paragraph should be indented.
%    \begin{macrocode}
\RequirePackage{indentfirst}
%    \end{macrocode}
%\section{Configuration of selected packages}
% Configuration of typical packages used when creating a diploma thesis in computer science.
%    \begin{macrocode}
    \AtBeginDocument{
% The 'algorithm2e' package
        \@ifpackageloaded{algorithm2e}{        
            \ifenglish\else
                \renewcommand{\listalgorithmcfname}{Lista algorytmów}%
                \renewcommand{\algorithmcfname}{Algorytm}%
            \fi
        }{}
% The 'minted' package
        \@ifpackageloaded{minted}{
            \ifenglish\else
                \SetupFloatingEnvironment{listing}{name=Kod źródłowy}
                \SetupFloatingEnvironment{listing}{listname=Lista kodów źródłowych}
            \fi
            \setminted{
% Display line numbers                
                linenos,     
% A single frame around the listing                
                frame=single
            }
        }{}
%        The 'listings' package
        \@ifpackageloaded{listings}{
            \lstset{          
% Whether to underscore spaces inside strings                
                showstringspaces=false,    
% Where to place line numbers                     
                numbers=left,  
% Step between two line numbers. If 1, each line will be numbered                                 
                stepnumber=1,      
% How far the line numbers are offset from the code                             
                numbersep=5pt,  
% Whether to show spaces within strings using underscores                                
                showspaces=false,     
% Whether to show tabs within strings using underscores                          
                showtabs=false,  
% Add a single border around the code                               
                frame=single,  
% Set the default tab size to two spaces                                
                tabsize=2, 
% Set position for caption - captions on top                                    
                captionpos=t,
% Enable automatic line breaks                                   
                breaklines=true, 
% Should automatic breaking only occur on white characters?                               
                breakatwhitespace=false,   
% Color and font for keywords                     
                keywordstyle=\ttfamily\color{blue}, 
% Color and font identifiers                
                identifierstyle=\ttfamily\color{violet}\bfseries, 
% Color and font for comments                
                commentstyle=\color{brown},   
% Color and font for strings
                stringstyle=\ttfamily,         
% If you want to add LaTeX within your code                
                escapeinside={\%*}{*)},         
                inputencoding=utf8,     
% The listings package does not support Polish UTF-8 characters, but this can be "fixed" - Polish characters will be replaced with LaTeX commands - see below                        
                literate={ą}{{\k{a}}}1
                    {Ą}{{\k{A}}}1
                    {ę}{{\k{e}}}1
                    {Ę}{{\k{E}}}1
                    {ó}{{\'o}}1
                    {Ó}{{\'O}}1
                    {ś}{{\'s}}1
                    {Ś}{{\'S}}1
                    {ł}{{\l{}}}1
                    {Ł}{{\L{}}}1
                    {ż}{{\.z}}1
                    {Ż}{{\.Z}}1
                    {ź}{{\'z}}1
                    {Ź}{{\'Z}}1
                    {ć}{{\'c}}1
                    {Ć}{{\'C}}1
                    {ń}{{\'n}}1
                    {Ń}{{\'N}}1
            }
            \ifenglish
                \renewcommand{\lstlistlistingname}{List of Listings}
            \else
                \renewcommand{\lstlistingname}{Kod źródłowy}%
                \renewcommand{\lstlistlistingname}{Lista kodów źródłowych}
            \fi
        }{}
    }
%    \end{macrocode}

\definecolor{AGH@black}{RGB}{30,30,30}
\definecolor{AGH@red}{RGB}{167,25,48}
\definecolor{AGH@green}{RGB}{0,105,60}

% \section{AGH logo}
% Tikz code generated from SVG by 'svg2tikz' tool.
%    \begin{macrocode}
\def\logoscale {1.000000}
\tikzset {
% AGH logo
    logoAGH/.pic= {
            \begin{tikzpicture}[y=1cm, x=1cm, yscale=\logoscale,xscale=\logoscale, every node/.append style={scale=\logoscale}, inner sep=0pt, outer sep=0pt]
                \begin{scope}[cm={ 1.25,-0.0,-0.0,-1.25,(0.0, 6.288)}]
                    \begin{scope}[shift={(-0.425\textwidth,10)}]
                        \path[fill=AGH@black,nonzero rule] (7.2964, -4.8072) -- (7.184, -4.8072) -- (7.2394, -4.9752) -- (7.2964, -4.8072) -- cycle(7.3638, -4.6082) -- (7.5038, -4.6082).. controls (7.5038, -4.6082) and (7.3489, -5.044) .. (7.3488, -5.0441).. controls (7.3314, -5.0931) and (7.2939, -5.1006) .. (7.2818, -5.1006) -- (7.1465, -5.1006).. controls (7.1673, -5.0837) and (7.1745, -5.0726) .. (7.1745, -5.0556).. controls (7.1745, -5.0453) and (7.1722, -5.0374) .. (7.1609, -5.004) -- (7.0269, -4.6082) -- (7.1184, -4.6082) -- (7.1589, -4.731) -- (7.3222, -4.731) -- (7.3638, -4.6082) -- cycle;
                        \path[fill=AGH@black,nonzero rule] (7.9315, -4.678) -- (7.9315, -4.8351).. controls (7.9315, -4.8651) and (7.9251, -4.8805) .. (7.9038, -4.8995) -- (8.0049, -4.8995).. controls (8.0362, -4.8995) and (8.0629, -4.8742) .. (8.0629, -4.8422) -- (8.0629, -4.6193).. controls (8.0135, -4.6048) and (7.9634, -4.5968) .. (7.9177, -4.5968).. controls (7.7427, -4.5968) and (7.6317, -4.6989) .. (7.6317, -4.8499).. controls (7.6317, -4.9974) and (7.7429, -5.1123) .. (7.9201, -5.1123).. controls (7.9604, -5.1123) and (7.9987, -5.1052) .. (8.0335, -5.0927) -- (8.0335, -4.9941).. controls (8.0023, -5.0146) and (7.963, -5.0269) .. (7.9201, -5.0269).. controls (7.8351, -5.0269) and (7.7682, -4.9683) .. (7.7682, -4.8601).. controls (7.7682, -4.7454) and (7.8189, -4.6751) .. (7.9003, -4.6751).. controls (7.9105, -4.6751) and (7.9206, -4.6758) .. (7.9315, -4.678);
                        \path[fill=AGH@black,nonzero rule] (8.4136, -5.037).. controls (8.4136, -5.0722) and (8.3858, -5.1006) .. (8.3517, -5.1006) -- (8.2537, -5.1006).. controls (8.275, -5.0817) and (8.2814, -5.0662) .. (8.2814, -5.0362) -- (8.2814, -4.6082) -- (8.4136, -4.6082) -- (8.4136, -4.8153) -- (8.5639, -4.8153) -- (8.5639, -4.6082) -- (8.6957, -4.6082) -- (8.6957, -5.0992) -- (8.5639, -5.0992) -- (8.5639, -4.8988) -- (8.4136, -4.8988) -- (8.4136, -5.037) -- cycle;
                        \path[fill=AGH@red,nonzero rule] (8.217, -5.4289).. controls (8.3264, -5.5041) and (8.4221, -5.6204) .. (8.4221, -5.8666) -- (8.4221, -7.3439) -- (8.5315, -7.3439) -- (8.5315, -5.8666).. controls (8.5315, -5.5931) and (8.3948, -5.4836) .. (8.217, -5.4289);
                        \path[fill=AGH@red,nonzero rule] (8.3059, -5.4289).. controls (8.4085, -5.4768) and (8.5863, -5.5657) .. (8.5863, -5.8666) -- (8.5863, -7.3439) -- (8.6957, -7.3439) -- (8.6957, -5.8666).. controls (8.6957, -5.4973) and (8.4085, -5.4426) .. (8.3059, -5.4289);
                        \path[fill=AGH@red,nonzero rule] (8.1486, -5.4289).. controls (8.2302, -5.511) and (8.258, -5.6751) .. (8.258, -5.894) -- (8.258, -7.3439) -- (8.3674, -7.3439) -- (8.3674, -5.8666).. controls (8.3674, -5.6211) and (8.2853, -5.4973) .. (8.1486, -5.4289);
                        \path[fill=AGH@black,nonzero rule] (7.6151, -7.8363).. controls (7.7246, -7.7611) and (7.8203, -7.6448) .. (7.8203, -7.3986) -- (7.8203, -5.7025) -- (7.9297, -5.7025) -- (7.9297, -7.3986).. controls (7.9297, -7.6722) and (7.7929, -7.7816) .. (7.6151, -7.8363);
                        \path[fill=AGH@black,nonzero rule] (7.7041, -7.8363).. controls (7.8067, -7.7885) and (7.9845, -7.6996) .. (7.9845, -7.3986) -- (7.9845, -5.7025) -- (8.0939, -5.7025) -- (8.0939, -7.3986).. controls (8.0939, -7.7679) and (7.8067, -7.8227) .. (7.7041, -7.8363);
                        \path[fill=AGH@black,nonzero rule] (7.5467, -7.8363).. controls (7.6284, -7.7543) and (7.6561, -7.5901) .. (7.6561, -7.3713) -- (7.6561, -5.7025) -- (7.7656, -5.7025) -- (7.7656, -7.3986).. controls (7.7656, -7.6441) and (7.6835, -7.7679) .. (7.5467, -7.8363);
                        \path[fill=AGH@green,nonzero rule] (7.5331, -5.4289).. controls (7.4236, -5.5041) and (7.3279, -5.6204) .. (7.3279, -5.8666) -- (7.3279, -7.3439) -- (7.2184, -7.3439) -- (7.2184, -5.8666).. controls (7.2184, -5.5931) and (7.3552, -5.4836) .. (7.5331, -5.4289);
                        \path[fill=AGH@green,nonzero rule] (7.4441, -5.4289).. controls (7.3415, -5.4768) and (7.1637, -5.5657) .. (7.1637, -5.8666) -- (7.1637, -7.3439) -- (7.0542, -7.3439) -- (7.0542, -5.8666).. controls (7.0542, -5.4973) and (7.3415, -5.4426) .. (7.4441, -5.4289);
                        \path[fill=AGH@green,nonzero rule] (7.6015, -5.4289).. controls (7.5198, -5.511) and (7.492, -5.6751) .. (7.492, -5.894) -- (7.492, -7.3439) -- (7.3826, -7.3439) -- (7.3826, -5.8666).. controls (7.3826, -5.6211) and (7.4647, -5.4973) .. (7.6015, -5.4289);
                    \end{scope}
                \end{scope}
            \end{tikzpicture}
        }
}
%    \end{macrocode}
% \section{Commands}
%\begin{macro}{\supervisor}
%\changes{v1.0}{2024/02/08}{Initial version}
%    \begin{macrocode}
\newcommand{\supervisor}[1]{\gdef\@supervisor{#1}}
%    \end{macrocode}
%\end{macro} 
%
%\begin{macro}{\titlePL}
%\changes{v1.0}{2024/02/08}{Initial version}
%    \begin{macrocode}
\newcommand{\titlePL}[1]{\gdef\@titlePL{#1}}
%    \end{macrocode}
%\end{macro} 
%
%\begin{macro}{\titleEN}
%\changes{v1.0}{2024/02/08}{Initial version}
%    \begin{macrocode}
\newcommand{\titleEN}[1]{\gdef\@titleEN{#1}}
%    \end{macrocode}
%\end{macro} 
%
%\begin{macro}{\maketitle}
%\changes{v1.0}{2024/02/08}{Initial version}
%    \begin{macrocode}
\renewcommand\maketitle{
    \KOMAoptions{twoside = false}
    \begin{titlepage}
    \centering
            \tikz[overlay,transform shape,transform canvas,x=0.80pt,y=0.80pt,yscale=-1.000000]
            \path pic {logoAGH};\par
            \vspace*{3\baselineskip}%
            {    \bfseries\large
                Akademia Górniczo-Hutnicza
                im. Stanisława Staszica
                w Krakowie
            }\par
            \vspace*{\baselineskip}%
            {
                \large Wydział Informatyki
            }\par
            \vspace*{3\baselineskip}%
            {
                \scshape\Large Praca dyplomowa
            }\par
            \vspace*{3\baselineskip}%
            \ifenglish
            {\sffamily\Large\bfseries \@titleEN}\par%
            \vspace*{\baselineskip}%
            {\sffamily\small\bfseries \@titlePL}\par%
            \else
            {\sffamily\Large\bfseries \@titlePL}\par%
            \vspace*{\baselineskip}%
            {\sffamily\small\bfseries \@titleEN}\par%
            \fi
            \vfill
            \par
            \begin{tabular}{ll}
                Autor:         & {\bfseries\@author}      \\[0.8mm]
                Kierunek:      & {\bfseries\fieldOfStudy} \\[0.8mm]
                Opiekun pracy: & {\bfseries\@supervisor}  \\[0.8mm]
            \end{tabular}
            \par
        \vspace*{3\baselineskip}%
        \centering Kraków, \the\year
    \end{titlepage}%
    \KOMAoptions{twoside = true}
}
%    \end{macrocode}
%\end{macro}
% \section{Environments}
%\begin{environment}{abstractPL}
%\changes{v1.0}{2024/02/08}{Initial version}
%    \begin{macrocode}   
\newenvironment{abstractPL}{%
    \titlepage
    \null\vfil
    \@beginparpenalty\@lowpenalty
    \begin{center}%
        \bfseries Streszczenie
        \@endparpenalty\@M
    \end{center}}%
{\par\vfil\null\endtitlepage}
%    \end{macrocode}
%\end{environment}
%\begin{environment}{abstractEN}
%\changes{v1.0}{2024/02/08}{Initial version}
%    \begin{macrocode}
\newenvironment{abstractEN}{%
    \titlepage
    \null\vfil
    \@beginparpenalty\@lowpenalty
    \begin{center}%
        \bfseries Abstract
        \@endparpenalty\@M
    \end{center}}%
{\par\vfil\null\endtitlepage}
%    \end{macrocode}
%\end{environment} 
% \section{Modifications}
%  Prepend an "appendix" to the number of chapter entries of the appendix in the table of contents
%    \begin{macrocode}
\newcommand*{\tocchapapp}{}
\newcommand*{\appendixmore}{
    \addtocontents{\csname ext@toc\endcsname}{
        \string\DeclareTOCStyleEntry[numwidth+=5em,
            entrynumberformat=\string\appendixprefixformat]{chapter}{chapter}
    }
}
\newcommand*{\appendixprefixformat}[1]{\appendixname\space#1}
%    \end{macrocode}
%    \begin{macrocode}
%</class>
%    \end{macrocode}
%\Finale
